\documentclass{article}
\usepackage{graphicx}
\usepackage{float}
\usepackage{amsmath}
\usepackage{amssymb}

\begin{document}

\title{Technical Questions and Answers in Machine Learning}
\author{ONI Olalekan Joseph}

\maketitle

\section{Problem 2 **COMPLETED**}
Determine the first and second derivative with respect to $x$ of: $f(x)= \frac{1}{1 + e^{-x}} $

\subsection{Solution to Problem 2 }
 First Derivative: \\
 $$f(x) = (1 + e^-x)^{-1}$$ \\
 Using Chain Rule \\
 $$f'(x) = -1(1 + e^{-x})^{-2} \times -1e^{-x}$$ \\
 $$f'(x) = \frac{e^{-x}}{(1 + e^{-x})^{2}}$$ \\ \\
 
\noindent Second Derivative:  \\
 $$f'(x) = \frac{e^{-x}}{(1 + e^{-x})^{2}}$$ \\
 Using Quotient Rule of Differentiation \\ \\
 $g(x) = e^{-x}$  \qquad  $h(x) = (1 + e^{-x})^{2} $ \\\\
 $ g'(x) = -e^{x} $ \qquad $h'(x) = -2e^{-x}(1 + e^{-x}) $ \\
 $$f''(x) = \frac{h \times g' - g \times h' }{g^2} $$ \\
 $$f''(x) = \frac{-e^{-x} \times (1 + e^{-x})^{2} \enspace + \enspace 2e^{-2x} \times (1 + e^{-x})}{e^{-2x}} $$\\
 $$f''(x) = \frac{e^{-2x} - 1}{e^{-x}} $$
 
\section{Problem 3 **COMPLETED**}
 If I break a stick of unit length into three random pieces, what is the expected length of the largest piece? (You may need to state the assumptions that you make.)
 
\subsection{Solution to Problem 3}
Let the stick be broken at two points $X$ and $Y$. \\
Therefore, we have two independent random variables $X$, $Y$ which are both uniform in [0,1]. \\
Let $A=min(X,Y)$, $B=max(X,Y)$ and $C=max(A,1-B,B-A)$. \\ \\
Let $f_{C}(a)$ be the probability density function (pdf) of C and $ F_{c}(a)$ be the cummulative distribution function (cdf). Then: 

$$ F_{c}(a) = P(C \le a) = P(A \le a, 1-B \le a, B-A \le a)$$
The cdf for the unit square is then equivalent to: \\\\
\[F_{c}(a) = \left\{
\begin{array}{lr}
(3a-1)^2 & : \frac{1}{3} \le a \le \frac{1}{2} \\\\
1-3(1-a)^2 & : \frac{1}{2} \le a \le 1
\end{array}
\right.
\]

Then the pdf is: \\\\
\[f_{c}(a) = \left\{
\begin{array}{lr}
6(3a-1) & : \frac{1}{3} \le a \le \frac{1}{2} \\\\
6(1-a) & : \frac{1}{2} \le a \le 1
\end{array}
\right.
\]
\\
Therefore, the expected length of the largest piece (C) is: \\\\
$$\int_{\frac{1}{2}}^{\frac{1}{3}} 6a(3a-1)da + \int_{1}^{\frac{1}{2}} 6a(1-a)da = \frac{11}{18}$$


\section{Problem 8 **COMPLETED**}
What are the values of the constants $a$, $b$ and $c$ if one writes the following expression in the form: $ a(x - b)^{2} + c$ \\ 

 \begin{equation}\label{key}
 3x^{2} - 4x + 5
 \end{equation}
 
\subsection{Solution to Problem 8}
$$ 3(x^2 - \frac{4}{3}x + \frac{5}{3})$$ \\
$$ 3 \Big[ (x - \frac{2}{3})^2 -\frac{4}{9} + \frac{5}{3} \Big] $$ \\
$$ 3 \Big[ (x - \frac{2}{3})^2 + \frac{11}{9} \Big]  $$ \\
$$ 3 \big(x - \frac{2}{3}\big)^2 + \frac{11}{3}  $$ \\
$ a =3; \enspace b= \frac{2}{3}; \enspace c= \frac{11}{3} $

\section{Problem 6 **COMPLETED**}
A factory that makes light bulbs contains three machines. The machines manufacture 20\%, 30\% and 50\% of the total production. From their production, 5\%, 4\%, and 2\% respectively are faulty. I choose a collection of light bulbs at random from the output.

\subsection{Solution to Problem 6a}
If the collection contains two faulty light bulbs, what is the probability that they come from the same machine? \\\\
Let $P_{M_{Af}}$ represent Probability of faulty bulbs produced from the first machine (A). \\ \\
Similarly for second machine and third machine we have $P_{M_{Bf}}$ and $P_{M_{Cf}}$ respectively. \\ \\
$P_{M_{Af}} = \frac{5}{20}$; \enspace $P_{M_{Bf}} = \frac{4}{30}$; \enspace $P_{M_{Cf}} = \frac{2}{50}$. \\ \\

Let the probability that two faulty light bulbs from a collection come from the same machines be $P_{M_{2f}}$. \\\\
$$P_{M_{2f}} = \Big( \frac{5}{20} \times \frac{5}{20}\Big) + \Big( \frac{4}{30} \times \frac{4}{30}\Big) + \Big( \frac{2}{50} \times \frac{2}{50}\Big) = 0.00163$$

\subsection{Solution to Problem 6b}
Let the probability that the three faulty light bulbs from a collection come from the different machines be $P_{M_{3f}}$. \\\\
$$P_{M_{3f}} = \Big( \frac{5}{20} \times \frac{4}{30} \times \frac{2}{50} \Big) = \frac{1}{750}$$

\section{Problem 4 **COMPLETED**}
The twenty-first century began on 1 January 2001 (a Monday) and will end on 31 December 2100 (a Friday). What percentage of twenty-first century Wednesdays fall on the last day of a month? (This question requires some coding.)

\subsection{Solution to Problem 4}
The Python code snippet is shown in Figure 1. Percentage of the 21st century Wednesdays that fall on the last day of the month: \\\\
$$ = \frac{172}{5242} \times 100 \% $$
$$ 3.3\%  \textit{(1dp)}$$

\begin{figure}[H]
	\centering
	\includegraphics[width=1.2\linewidth, height=0.6\textheight]{pythonCode2}
	\caption{Python Source Code Screenshot}
	\label{fig:pythoncode}
\end{figure}

\section{Problem 1 **COMPLETED**}
Each night, a princess is equally likely to sleep on anything from six to twelve mattresses. On half of the nights of the year, a pea is placed underneath the lowest mattress. She never falls asleep if a pea is placed underneath a pile of six mattresses. If the pea is placed underneath seven mattresses, she sleeps wonderfully one night out of ten; under eight mattresses, she sleeps well two out of ten nights; and so on, until if the pea is placed underneath the full twelve mattresses, she sleeps well six out of ten nights. One morning, when her good friend Bayes woke her up, she said that she had slept incredibly well that night! What is the expected number of mattresses upon which she slept? \\\\
\subsection{Solution to Problem 1}
The expected value (or mean) of $X$, where $X$ is the expected number of mattresses is a discrete random variable. Let the expected value of $X$ be written as $E(X)$. \\\\


\begin{tabular}{|c|c|c|c|c|c|c|c|}
	\hline 
	Number of mattresses, S& 6 &7  & 8 & 9 &10  &11  & 12 \\ 
	\hline 
	isAsleep?& no & yes & yes & yes & yes & yes & yes \\ 
	\hline 
	Probability of good sleep with pea, $P(X=x)$& 0 & $\frac{1}{10}$ & $\frac{2}{10}$ & $\frac{3}{10}$ & $\frac{4}{10}$ & $\frac{5}{10}$ & $\frac{6}{10}$ \\ 
	\hline 
\end{tabular} \\\\

$ E(X) = S \times P(X = x) $ \\ \\
$ = 6 \times P(X=6) + 7 \times P(X=7) + 8 \times P(X=8) + 9 \times P(X=9) + 10 \times P(X=10) + 11 \times P(X=11) + 12 \times P(X=12)$ \\ \\
$$= 0 + \frac{7}{10} + \frac{16}{10} + \frac{36}{10} + \frac{40}{10} + \frac{55}{10} + \frac{72}{10}$$ \\
$$=22.6$$

\section{Problem 9}
Consider $n$ chords on a circle, each defined by its endpoints. Describe an $O(nlogn)$-time algorithm to determine the number of pairs of chords that intersect inside the circle. (For example, if the $n$ chords are all diameters that meet at the center, then the correct answer is $\binom n2$). Assume that no two chords share an endpoint.

\subsection{Solution to Problem 9}

\begin{itemize}
	\item Choose a point on the circle and assign it key value 1. The rank of any point will be its position when read in the sequence starting with point 1, and reading clockwise around the circle.
	\item  Next, we label each point with a key. Reading clockwise around the circle.
	\begin{itemize}
		\item If a point’s chordal companion has a key, assign it
		the same key.
		\item Otherwise, assign it the next lowest unused integer key value. 
	\end{itemize} 
	\item Then use the algorithm given in Listing 1 to count the
	number of inversions (based on key value), with the caveat that an inversion from key $i$ to key $j$ (with $i > j$) doesn’t count if the rank of the companion of $i$ is smaller than the rank of $j$.
\end{itemize}

\section{Problem 7}
If $\boldsymbol{w} \in \mathbb{R}^{d}$ and $\boldsymbol{x} \in \mathbb{R}^{d}$, and $f(\boldsymbol{w}) = sin(\boldsymbol{x}^{T}\boldsymbol{w})$, find expressions for $\frac{\partial f(\boldsymbol{w})}{\partial \boldsymbol{w}}$ and $\frac{\partial^{2} f(\boldsymbol{w})}{\partial \boldsymbol{w} \partial \boldsymbol{w}^{T}}$.

\subsection{Solution to Problem 7}
$$\frac{\partial f(\boldsymbol{w})}{\partial \boldsymbol{w}} = x^{T}\cos(x^{T}w)$$

\end{document}